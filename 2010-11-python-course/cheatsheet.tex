\documentclass[10pt]{article}

\usepackage{textcomp}
\usepackage[pdfborder={0,0,0}]{hyperref}
\usepackage{xcolor}
\usepackage{multicol}
\usepackage{mdwlist}
\usepackage{upquote}
\usepackage[utf8]{inputenc}
\usepackage{hyperref}

\hypersetup{pdfauthor=Petr Viktorin}

\usepackage{tgheros}
\renewcommand*\familydefault{\sfdefault}

\usepackage[top=0.5cm,left=0.5cm,right=0.5cm,bottom=0.5cm]{geometry}

\newcommand\x{$x$}
\newcommand\y{$x$}
\newcommand\z{$x$}
\renewcommand\t[1]{\par\hspace{0ex}\rlap{\tt #1}\hspace{3.5cm}}
\newcommand\n{}

\definecolor{silver}{rgb}{0.9999,0.9999,0.9999}
\def\<#1>{{\tt#1}}

\newcommand\magic[1]{\_\_#1\_\_}

\setlength\parindent{0pt}

\newenvironment{items}{\begin{itemize*}\setlength\itemsep{0pt}\setlength\parskip{0pt}\setlength\parsep{0pt}}{\end{itemize*}}

\begin{document}

\section{Statements}

\subsection{Simple statements}

\t{a = b} Assign the value of \<b> to the name \<a>.
\t{a, b, c = t = e} Assign the value of \<e> to \<t>, and its 3 elements to \<a>, \<b>, and \<c>.
\t{a += 1} Add 1 to \<a>. Similarly \<-=>, \<*=>, \</=>, \<//=>, \<\%>, \<**=>, \<{>>=}>, \<{<<=}> \<\&=> \<\textasciicircum=>, \<|=>
\t{assert x} Raise an AssertionError if x is false
\t{pass} Do nothing
\t{del x} Remove the variable \<x>. Similarly \<del a.b> for removing attributes, \<del x[k]>, etc
\t{raise e} Raise the exception \<e>.
\t{import m} Import the module \<m>, assign it to the variable \<m>.
\t{from m import x} Import \<x> from the module \<m>, and assign it to the variable \<x>.
\t{from m import x as y} \hspace{1em} Import \<x> from the module \<m>, and assign it to the variable \<y>.
\t{global x} Mark \<x> as global; assignments to it will be made to the module-level variable.
\t{print x} Print out \<x>. (In Python 3, this becomes a function)
\t{exec x} Execute Python code given by \<x>. (In Python 3, this becomes a function)
\t{break} Terminate a for or while loop
\t{continue} Terminate the current pass through a for/while loop, go to the next one
\t{return x} Exit the current function, returning x
\t{return} Exit the current function, returning None
\\Also any expression is a statement.

\subsection{Compound statements}

\begin{multicols}{2}
\subsubsection{Branching: if}
\begin{verbatim}
if c1:
    print 'c1 was true'
elif c2:
    print 'c2 was true, c1 was not'
else:
    print 'neither was true'
\end{verbatim}\columnbreak
\subsubsection{Context: with}
The \verb+with+ statement is used to introduce \emph{context}.
For example with a file, it will be closed after the block is done.
\begin{verbatim}
with open('filename') as f:
    print f.read()
# f will be closed here
\end{verbatim}\end{multicols}


\subsubsection{Looping: for/while}
\begin{multicols}{2}
\begin{verbatim}
for x in some_list:
    print x

for key, value in dictionary.items():
    print key, value
\end{verbatim}\columnbreak
\begin{verbatim}
response = ask_for_input()
while response != 'yes':
    print 'You need to agree'
    response = ask_for_input()
\end{verbatim}\end{multicols}
The loop constructs may also have an \<else> clause, which executes after the loop is finished (if not terminated by break, return, raise, etc.)

\begin{multicols}{2}
\subsubsection{Exception handling: try/except/finally}
\begin{verbatim}
try:
    do_dangerous_stuff()
except KeyError, e:
    print "Key not found:", e
except Exception:
    print "Exception occured"
else:
    print "No exception occured"
finally:
    print "This executes either way"
\end{verbatim}\columnbreak
\subsubsection{Functions and Classes: def, class}
\begin{verbatim}
def func(argument1, argument2):
    return 'value'

def generator(argument, argument2)
    yield 'value'

class C(superclass):
    def __init__(self, argument1, argument2):
        print "Do initialization"
    def method(self, argument1, argument2):
        print "method called"
\end{verbatim}\end{multicols}

\pagebreak
\section{Operators}

\begin{multicols}{2}
\subsection{General}

\begin{items}
\item Boolean \<and> \<or> (short-circuiting), \<not>.
\item Comparison~~~\<{<}>~~~\<{>}>~~~\<{<=}>~~~\<{>=}>~~~\<==>~~~\<!=>,~~~\<is>,~~~\<is not>. \\You can use \<x < y < z>
\item Numeric~~~\<+>~~~\<->~~~\<*>~~~\</>~~~\<//>~~~\<\%>~~~\<**>
\item Containers \<in>, \<not in>
\item Bitwise \<|> \<\textasciicircum> \<\&> \<{<<}> \<{>>}> \<$\sim$>
\end{items}

\columnbreak
\subsection{Indexing}

\begin{items}
\item Sequences: \<li[i]> (with integers, negative=from end)
\item Dicts: \<d[k]> (with any hashable type)
\item Sequence slices: \<li[start:stop:step]> ([::] is [0:-1:1])
\item Deleting: \<del li[i]>, \<del li[i:j:s]>, \<del d[k]>
\item Replacing: \<li[i] = new\_value>
\item Replacing slices: \<li[i:j] = new\_sublist>
\end{items}
\end{multicols}
\subsection{Special operators}
\t{func(argument)} Call
\t{obj.attr} Attribute access
\t{(yield x)} Yield \<x> from a generator; the value is whatewer was passed to send()
\t{lambda p: e} Lambda “mini-function” (roughly equivalent to \<def \_func(p): return e>)
\t{x if c else y} Evaluate either \<x> or \<y>, depending on condition
\t{[f(x) for x in lst if c(x)]} \hspace{5em} List containing f(x) for each element x of the iterable lst that satisfies condition c(x)
\t{(f(x) for x in lst if c(x))} \hspace{5em} Same as above, but is an iterator instead of a list
\t{\textasciigrave x\textasciigrave} Equivalent to \<repr(x)>. Not recommended.


\section{Native Datatypes}

\subsection{Numeric types}

Immutable, hashable

\begin{multicols}{2}
\begin{itemize}\setlength\itemsep{0pt}\setlength\parskip{0pt}\setlength\parsep{0pt}
\item int: 12, 1234567890987654321, 0o711, 0x4f5
\item float: 1.2, 1.3e8
\item complex: 3+5j
\item long (large integers)
\par {\tt type(int(4e80)) is long}
\item bool: \<True> or \<False> (using as numbers is bad!)
\end{itemize}

\columnbreak
\t{abs(x)}
\\\<import math> for working with numbers, \<cmath> for complex
\t{math.sin(x)} \n \<math.cos(x)>
\t{math.log(x, base)} \<math.sqrt(x)>

\t{math.pi} \n \<math.e>

\end{multicols}

\subsection{Strings}

Immutable, hashable

\begin{itemize}\setlength\itemsep{0pt}\setlength\parskip{0pt}\setlength\parsep{0pt}
\item str: \<"abc">, \<\textquotesingle abc\textquotesingle>, \<"""abc""">, \<\textquotesingle\textquotesingle\textquotesingle abc\textquotesingle\textquotesingle\textquotesingle>
\item unicode: \<u"abc">, etc.
\end{itemize}

If using unicode characters in the source, specify an encoding in the first line:
\verb+# Encoding: UTF-8+

In Python 3, all strings are Unicode (unles given as \<b"abc">), and the default encoding is UTF-8.

Escaped characters (\<\textbackslash n>, \<\textbackslash t>, \<\textbackslash \textquotesingle>, \<\textbackslash ">, etc.)
are interpreted, unless the string is \verb+r+ (\<r"\textbackslash">)

\begin{multicols}{2}
\subsection{Interpolation}

\t{"..\%s.." \% value} \n Substitute str(value) into the string
\t{"\%s, \%s" \% (a, b)} \n Substitute str(a) and str(b)
\t{"\%.2f" \% 3.5} \n Formatting; see Python docs

\subsection{Formatting}
\verb|"{0} {c.imag}".format("abc", c=3+2j)|

\verb|"{val:.3f}".format(val=3.1415926)|

\verb|"{lst[1]}".format(lst=[0, 1, 2])|
\end{multicols}

\subsection{Dictionaries}

Dictionaries are mutable. Dictionary keys must be hashable.

Dictionary elements are not ordered.

\begin{itemize}\setlength\itemsep{0pt}\setlength\parskip{0pt}\setlength\parsep{0pt}
\item dict: \verb+{}+, \verb+{'one': 1, 'two': 2}+, \verb+dict(one=1, two=2)+, \verb+dict([('one', 1), ('two', 2)])+
\end{itemize}

\begin{multicols}{2}
\t{d.get(key, default=None)}
\t{d.clear()}
\t{d.has\_key(key)}
\t{d.setdefault(key, default=None)}
\par\verb}d.items()},
\verb}d.keys()},
\verb}d.values()}
\\\verb}d.iteritems()} etc.
\t{d.popitem()}
\t{d.update(other)}
\end{multicols}

\subsection{Sequences}

Tuples are immutable. Tuples are hashable iff they only contain hashable objects.

Lists are mutable.

\begin{itemize}\setlength\itemsep{0pt}\setlength\parskip{0pt}\setlength\parsep{0pt}
\item list: \<[]>, \<[1, 2, 3]>, \<[1, 2, 3, ]>
\item tuple:  \<()>, \<(1, )>, \<(1, 2, 3)>
\end{itemize}


\begin{multicols}{2}
\t{s + t} \n concatenation
\t{s * n, n * s} \n $n$ copies concatenated

\t{len(s)} \n
\t{min(s, key=f)}
\t{max(s, key=f)}
\t{sorted(s, key=f)} \n  sorted according to key function
\end{multicols}

\subsubsection{Mutable sequences}

\begin{multicols}{3}
\t{s.append(x)}
\t{s.extend(x)}
\t{s.count(x)}
\t{s.insert(i, x)}
\t{s.pop(i=-1)}
\t{s.remove(x)}
\t{s.reverse()}
\\{\tt s.sort(key=?, reverse=False)}
\\{\tt s.index(x, start, stop)}
\end{multicols}


A key function should take an item of the sequence as argument, and return the sort key (use a tuple for sorting by more criteria).
{\tt sort(people, key=lambda p: (p.lastName, p.firstName))}

\subsection{Sets}

Sets are mutable; frozensets are like tuples. Set elements must be hashable.

Set elements are not ordered.


\begin{itemize}\setlength\itemsep{0pt}\setlength\parskip{0pt}\setlength\parsep{0pt}
\item set: \<set([1, 2, 3])>
\item frozenset \<frozenset([1, 2, 3])>
\end{itemize}

\begin{multicols}{2}
\verb>s.isdisjoint(other)>
\t{s <= other} \n \verb+s.issubset(other)+
\t{s >= other} \n \verb+s.issuperset(other)+
\t{s | other | ...} \n \verb>s.union(other, ...)>
\t{s \& other \& ...} \n \verb>s.intersection(other, ...)>
\t{s - other - ...} \n \verb>s.difference(other, ...)>
\t{s {\textasciicircum} other} \n \verb>s.symmetric_difference(other)>

\columnbreak

\t{s |= other | ...} \n \verb>s.update(other, ...)>

similarly \&=, -=, \textasciicircum= (\verb>s.intersection_update(other)>, etc.)

\t{s.discard(elem)} Remove if exists
\t{s.pop()} \n Remove \& return arbitrary element

\verb>s.add(elem)>

\verb>s.remove(elem)>

\verb>s.clear()>
\end{multicols}

\subsection{Files}

Iterating over a file yields the lines in the file.

\t{open(filename, mode="r")}

Mode is "r" (read), "w" (write) or "a" (append); add "b" for binary.

\begin{multicols}{2}
\t{f.read()} \n Read everything
\t{f.read(size)} \n Read up to $size$ bytes
\t{f.readline()} \n Read a line
\t{f.close()} \n Close the file
\end{multicols}

\subsection{Iterators}
Any sequence, file, function with \<yield>, etc. may be iterated over. \<for> calls these functions automatically.
\t{it = iter(x)} \n Return an iterator over $x$
\t{it.next()} \n Return the next value in an iterator, or raise \<StopIteration>

\subsection{None}
Immutable, hashable

None is similar to NULL in C++/Java. It represents a missing value

It is the default return value of functions (missing \<return>, or \<return> with no value).

\pagebreak
\section{Magic Methods}
Define these in your own class to gain specific functionality.
\begin{multicols}{2}
\subsection{Basics}
\t{C.\magic{init}()} \n Initialize the object
\subsection{Conversion}
\t{c.\magic{str}()} \n \verb+str(c)+
\t{c.\magic{repr}()} \n \verb+repr(c)+
\t{c.\magic{format}(spec)} \n \verb+"{0:spec}".format(c)+
\t{c.\magic{complex}()} \n \verb+complex(c)+
\t{c.\magic{int}()} \n \verb+abs()+
\t{c.\magic{float}()} \n \verb+abs()+
\t{c.\magic{nonzero}()} \n Boolean value
\subsection{Iteration}
\t{c.\magic{iter}()} \n \verb+iter(c)+
\t{c.\magic{reversed}()} \n Reverse iterator
\subsection{Attribute access}
\t{c.\magic{getattr}(a)} \n \verb+c.a+, fallback
\t{c.\magic{getattribute}(a)} \n \hspace{1ex}\verb+c.a+, unconditional
\t{c.\magic{setattr}(a, v)} \n \verb+c.a = v+
\t{c.\magic{delattr}(a)} \n \verb+del c.a+
\t{c.\magic{dir}()} \n List attributes (for dir())
\subsection{Calling}
\t{c.\magic{call}(...)} \n \verb+c(...)+
\subsection{Emulating sets}
\t{c.\magic{contains}(e)} \n \verb+e in c+
\t{c.\magic{len}()} \n Length (for len())
\subsection{Emulating dictionaries}
\t{c.\magic{getitem}(x)} \n \verb+c[x]+
\t{c.\magic{setitem}(x, v)} \n \verb+c[x] = v+
\t{c.\magic{delitem}(x)} \n \verb+del c[x]+
\t{c.\magic{missing}(x)} \n \verb+c[x]+ if \verb+c not in x+
\subsection{Emulating numbers}
\t{c.\magic{add}(x)} \n \verb-c + x-
\t{c.\magic{sub}(x)} \n \verb+c - x+
\t{c.\magic{mul}(x)} \n \verb+c * x+
\t{c.\magic{div}(x)} \n \verb+c / x+
\t{c.\magic{mod}(x)} \n \verb+c % x+
\t{c.\magic{floordiv}(x)} \n \verb+c // x+
\t{c.\magic{divmod}(x)} \n \verb+divmod(c, x)+
\t{c.\magic{pow}(x)} \n \verb+c ** x+
\t{c.\magic{rshift}(x)} \n \verb+c >> x+
\t{c.\magic{lshift}(x)} \n \verb+c << x+
\t{c.\magic{and}(x)} \n \verb+c & x+
\t{c.\magic{xor}(x)} \n \verb+c ^ x+
\t{c.\magic{or}(x)} \n \verb+c | x+
\smallskip
\t{c.\magic{radd}(x)} \n \verb-x + c-
\par etc.
\smallskip
\t{c.\magic{iadd}(x)} \n \verb-c += x-
\par etc.
\smallskip
\t{c.\magic{neg}(x)} \n \verb+-c+
\t{c.\magic{pos}()} \n \verb-+c-
\t{c.\magic{abs}()} \n \verb+abs(c)+
\t{c.\magic{invert}()} \n \verb+~c+
\t{c.\magic{round}(n)} \n \verb+round(c, n)+
\t{c.\magic{index}()} \n \verb+lst[c]+
\t{c.\magic{oct}()} \n \verb+oct(c)+
\t{c.\magic{hex}()} \n \verb+hex(c)+
\t{c.\magic{coerce}(o)} \n (c, o) converted to common type
\subsection{Comparisons}
\t{c.\magic{lt}(o)} \n \verb+c < o+
\t{c.\magic{gt}(o)} \n \verb+c > o+
\t{c.\magic{le}(o)} \n \verb+c <= o+
\t{c.\magic{ge}(o)} \n \verb+c >= o+
\t{c.\magic{eq}(o)} \n \verb+c == o+
\t{c.\magic{ne}(o)} \n \verb+c != o+
\subsection{Context handlers}
\t{c.\magic{enter}(o)} \n \verb+with c:+ (entry)
\t{c.\magic{exit}(o)} \n \verb+with c:+ (exit)
\subsection{Low-level stuff}
Read the documentation!
\t{C.\magic{new}(params)} \n Override object construction
\t{c.\magic{del}()} \n Clean up object
\t{C.\magic{slots}} \n Limit attributes
\t{c.\magic{hash}()} \n Compute hash value
\t{C.\magic{metaclass}} \n The class of the class
\t{C.\magic{instancecheck}(x)} \n \hspace{1em}\verb+isinstance(x, C)+
\t{C.\magic{subclasscheck}(X)} \n \hspace{1em}\verb+issubclass(X, C)+
\end{multicols}

\pagebreak
\section{Good practice}

\begin{items}
\item Name your variables consistently.
\item Don't use names of built-in objects.
\item Indent each block with four spaces.
\item Write good docstrings.
\begin{multicols}{3}\small
\colorbox{silver}{\rlap{Bad}\hspace{0.8\columnwidth}}
\begin{verbatim}
def factorial(n):
    if n > 1:
        return n * factorial(n - 1)
    else:
        return 1
\end{verbatim}
\columnbreak
\colorbox{silver}{\rlap{Better}\hspace{0.8\columnwidth}}
\begin{verbatim}
def factorial(n):
    "compute factorial"
    if n > 1:
        return n * factorial(n - 1)
    else:
        return 1
\end{verbatim}
\columnbreak
\colorbox{silver}{\rlap{Best}\hspace{0.8\columnwidth}}
\begin{verbatim}
def factorial(n):
    """Return the factorial of n

    n: nonnegative integer
    """
    if n > 1:
        return n * factorial(n - 1)
    else:
        return 1
\end{verbatim}
\end{multicols}

\item Use language features that apply. For example, iterate over anything iterable.
\begin{multicols}{3}\small
\colorbox{silver}{\rlap{Bad}\hspace{0.8\columnwidth}}
\begin{verbatim}
result = []
i = 0
while i < len(lst):
    result.append(lst[i] * 2)
    i += 1
\end{verbatim}
\columnbreak
\colorbox{silver}{\rlap{Better}\hspace{0.8\columnwidth}}
\begin{verbatim}
result = []
for item in lst:
    result.append(item * 2)
\end{verbatim}
\columnbreak
\colorbox{silver}{\rlap{Best}\hspace{0.8\columnwidth}}
\begin{verbatim}
result = [item * 2 for item in lst]
\end{verbatim}
\end{multicols}

\item It's better to beg for forgiveness than to ask for permission.
\begin{multicols}{3}\small
\colorbox{silver}{\rlap{Bad}\hspace{0.8\columnwidth}}
\begin{verbatim}
if 'key' in dictionary:
    return dictionary[key]
else:
    return 'default value'
\end{verbatim}
\columnbreak
\colorbox{silver}{\rlap{Better}\hspace{0.8\columnwidth}}
\begin{verbatim}
try:
    return dictionary[key]
except Exception:
    return 'default value'
\end{verbatim}
\columnbreak
\colorbox{silver}{\rlap{Best}\hspace{0.8\columnwidth}}
\begin{verbatim}
return dictionary.get(
        key, 'default value')
\end{verbatim}
\end{multicols}

\item Numeric zeroes and empty containers are false
\begin{multicols}{3}\small
\colorbox{silver}{\rlap{Bad}\hspace{0.8\columnwidth}}
\begin{verbatim}
if length(lst) == 0:
    print 'Empty'
\end{verbatim}
\columnbreak
\colorbox{silver}{\rlap{Better}\hspace{0.8\columnwidth}}
\begin{verbatim}
if length(lst):
    print 'Empty'
\end{verbatim}
\columnbreak
\colorbox{silver}{\rlap{Best}\hspace{0.8\columnwidth}}
\begin{verbatim}
if not lst:
    print 'Empty'
\end{verbatim}
\columnbreak
\end{multicols}

\item Use the standard library!
\begin{multicols}{2}\small
\colorbox{silver}{\rlap{Bad}\hspace{0.8\columnwidth}}
\begin{verbatim}
def read_csv_file(file):
    for line in file:
        yield line.split(',')

for record in read_csv_file(open('data.csv')):
    # process record
\end{verbatim}
\columnbreak
\colorbox{silver}{\rlap{Better}\hspace{0.8\columnwidth}}
\begin{verbatim}
import csv

for record in csv.reader(open('data.csv')):
    # process record
\end{verbatim}
\columnbreak
\end{multicols}

\item Follow PEP8 to make your code look perfect.
\\\url{http://www.python.org/dev/peps/pep-0008/}
\end{items}



\vfill\footnotesize
Python Cheatsheet by Petr Viktorin \href{mailto:encukou@gmail.com}{\texttt{<encukou@gmail.com>}}

This work is licensed under a \href{http://creativecommons.org/licenses/by-sa/3.0/}{Creative Commons Attribution-ShareAlike 3.0 Unported License}.

\end{document}


