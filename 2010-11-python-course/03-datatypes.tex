\documentclass[reference]{pyslides}

\title{Native Datatypes}
\pyslidenumber{3}
\date{November 2010}

%%%%%%%%%%%%%%%%%%%%%%%%%%%%%%%%%%%%%%%%%%%%%%%%%%%
\begin{document}

\begin{frame}\titlepage\end{frame}

\section{}

\begin{frame}[fragile]{Built-In Types}
One of Python's biggest strengths are its built-in data types. We will cover the important ones here:
\begin{itemize}
\item Numbers
\item Lists
\item Tuples
\item Dictionaries
\item Strings
\end{itemize}
\end{frame}

\section{Numbers}

\begin{frame}[fragile]{Numbers}
Python provides integers, floating-point numbers and complex numbers.
\input "|./highlight.py samples/03numbers1.txt fontsize=!small,lastline=14 no"
\end{frame}

\begin{frame}[fragile]{Number Gotchas}
Be careful when dividing two integers: the result is an integer, so the answer will be rounded down.
\input "|./highlight.py samples/03numbers1.txt fontsize=!small,firstline=16,lastline=19 no"
(Fixed in Python 3, use \verb+//+ if you need this behavior.)

\bigskip

If you work with floats, you might get inexact results.
\input "|./highlight.py samples/03numbers1.txt fontsize=!small,firstline=21,lastline=999 no"

\bigskip

{\small (The standard library provides Fraction and Decimal numbers you can use to get around these limitations.)}
\end{frame}

\section{Lists}

\begin{frame}[fragile]{Defining Lists}
\input "|./highlight.py samples/03list1.txt fontsize=!small,lastline=12 no"
\end{frame}

\begin{frame}[fragile]{Slicing Lists}
\input "|./highlight.py samples/03list1.txt fontsize=!small,firstline=13,lastline=28 no"
\input "|./highlight.py samples/03listslice.txt fontsize=!small,firstline=2 no"
\end{frame}

\begin{frame}[fragile]{Adding to Lists}
\input "|./highlight.py samples/03list1.txt fontsize=!small,firstline=30,lastline=41 no"
\end{frame}

\begin{frame}[fragile]{Append vs. Extend}
\input "|./highlight.py samples/03list1.txt fontsize=!small,firstline=43,lastline=58 no"
\end{frame}

\begin{frame}[fragile]{Searching Lists}
\input "|./highlight.py samples/03list1.txt fontsize=!small,firstline=60,lastline=75 no"
\end{frame}

\begin{frame}[fragile]{Removing from Lists}
\input "|./highlight.py samples/03list1.txt fontsize=!small,firstline=76,lastline=90 no"
\end{frame}

\begin{frame}[fragile]{List Operators}
\input "|./highlight.py samples/03list1.txt fontsize=!small,firstline=92,lastline=280 no"
\end{frame}

\section{Tuples}

\begin{frame}[fragile]{Defining Tuples}
\input "|./highlight.py samples/03tuple1.txt fontsize=!small,lastline=11 no"
\end{frame}

\begin{frame}[fragile]{Tuples are Immutable}
\input "|./highlight.py samples/03tuple1.txt fontsize=!small,firstline=13,lastline=23 no"
\end{frame}

\begin{frame}[fragile]{Assigning Tuples to Variables}
\input "|./highlight.py samples/03tuple1.txt fontsize=!small,firstline=24,lastline=32 no"
\end{frame}

\begin{frame}[fragile]{Parentheses are Optional}
\input "|./highlight.py samples/03tuple1.txt fontsize=!small,firstline=33,lastline=41 no"
\end{frame}

\begin{frame}[fragile]{Returning Two Values}
\input "|./highlight.py samples/03return2.py fontsize=!small,lastline=4 no"
\end{frame}

\section{Dictionaries}

\begin{frame}[fragile]{Defining a Dictionary}
A dictionary is a data type that maps \emph{keys} to \emph{values}.
\input "|./highlight.py samples/03dict1.txt fontsize=!small,lastline=11 no"

\bigskip

Recent Python versions allow a shorthand syntax:
\input "|./highlight.py samples/03dict1.txt fontsize=!small,firstline=13,lastline=15 no"
\end{frame}

\begin{frame}[fragile]{Modifying a Dictionary}
\input "|./highlight.py samples/03dict1.txt fontsize=!small,firstline=16,lastline=25 no"

\end{frame}

\begin{frame}[fragile]{Mixing Datatypes in a Dictionary}
\input "|./highlight.py samples/03dict1.txt fontsize=!small,firstline=26,lastline=37 no"
\end{frame}

\begin{frame}[fragile]{Mixing Datatypes in a Dictionary: Limitations}
The \emph{keys} in a dictionary must be \emph{hashable}, that is, objects whose value cannot be changed.

\bigskip

This means you cannot use lists and dictionaries.

You can use tuples, as long they only contain hashable objects.

\input "|./highlight.py samples/03return2.py fontsize=!small,firstline=6 no"

\bigskip

The \emph{values} of a dictionary can be of any type.

\end{frame}

\begin{frame}[fragile]{Deleting from a Dictionary}
\input "|./highlight.py samples/03dict1.txt fontsize=!small,firstline=38,lastline=48 no"
\end{frame}

\begin{frame}[fragile]{Keys, Values, Items}
\input "|./highlight.py samples/03dict1.txt fontsize=!small,firstline=50 no"
\end{frame}


\section{Strings}

\begin{frame}[fragile]{Strings}
Strings are like tuples: immutable sequences.

Unlike tuples, they can only contain characters.

\input "|./highlight.py samples/03string1.txt fontsize=!small,lastline=6 no"

\bigskip

The characters are themselves strings. Watch out for that.
\end{frame}

\begin{frame}[fragile]{Defining Strings}
\input "|./highlight.py samples/03string-def.txt fontsize=!small no"
\end{frame}

\begin{frame}[fragile]{String Methods}
\input "|./highlight.py samples/03string1.txt fontsize=!small,firstline=8,lastline=24 no"
\end{frame}

\begin{frame}[fragile]{String Formatting}
\input "|./highlight.py samples/03string1.txt fontsize=!small,firstline=25,lastline=32 no"
\end{frame}

\begin{frame}[fragile]{String Formatting vs. Concatenation}
\input "|./highlight.py samples/03string1.txt fontsize=!small,firstline=34,lastline=48 no"
\end{frame}

\begin{frame}[fragile]{Formatting Numbers}
\input "|./highlight.py samples/03string1.txt fontsize=!small,firstline=50,lastline=62 no"
\end{frame}

\begin{frame}[fragile]{Joining and Splitting}
\input "|./highlight.py samples/03string1.txt fontsize=!small,firstline=64,lastline=999 no"
\end{frame}

\begin{frame}[fragile]{Unicode strings}
To use Unicode in your file, define an encoding in the \emph{first} or second line of the file.

\bigskip

\input "|./highlight.py samples/03unicode.txt fontsize=!small no"

\bigskip

{\small (In Python 3, all strings are unicode by default)}
\end{frame}

\section{Other types}

\begin{frame}[fragile]{None}
\input "|./highlight.py samples/03others1.txt fontsize=!small,lastline=11 no"
\end{frame}

\begin{frame}[fragile]{Booleans}
\input "|./highlight.py samples/03others1.txt fontsize=!small,firstline=12 no"
\end{frame}

\begin{frame}[fragile]{Sets}
Sets are made by from an iterable (list or dict) using the \verb+set()+ function.

Read about them in the Pyhon reference if you need them.
\end{frame}

\section{Working with types}

\begin{frame}[fragile]{Converting types}
The name of a type acts as a conversion function.
\input "|./highlight.py samples/03converting.txt fontsize=!small no"

\bigskip

Note the names Python uses.
\end{frame}

\begin{frame}[fragile]{List Comprehensions}
\input "|./highlight.py samples/03comprehension.txt fontsize=!small,firstline=1,lastline=8 no"
\end{frame}

\begin{frame}[fragile]{Putting It All Together}
\input "|./highlight.py samples/03comprehension.txt fontsize=!small,firstline=10,lastline=999 no"
\end{frame}

\begin{frame}[fragile]{Your First Python Program}
The registration program should now make perfect sense.

\bigskip

\input "|./highlight.py samples/02firstprogram.py fontsize=!tiny"

\bigskip
\end{frame}

\end{document}
