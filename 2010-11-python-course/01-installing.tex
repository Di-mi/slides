\documentclass{pyslides}

\title{Installing Python}
\pyslidenumber{1}
\date{November 2010}

\begin{document}

\begin{frame}\titlepage\end{frame}

\section{Course information}

\begin{frame}{Course information}
\begin{itemize}
\item Web page: \href{http://cs.joensuu.fi/~pviktor/python}{cs.joensuu.fi/{\texttildelow}pviktor/python}
\item E-mail: \href{mailto:pviktor@cs.joensuu.fi}{pviktor@cs.joensuu.fi}
\item 5 sessions of 2 hours each (Tuesday, 16:00--18:00)
\item 1--2 ECTS credits {\small (check if your program willl accept them!)}
\item Project {\small (more information later)}
\end{itemize}
\end{frame}

\section{What is Python?}

\begin{frame}{The Python programming language}
\begin{itemize}
\item A strongly and dynamically typed high-level multi-paradigm programming language
\item First published in 1991 by Guido van Rossum
\item Used for
        \href{http://www.blender.org/education-help/python/}{scripting},
        \href{http://www.reddit.com/}{Web} \href{http://youtube.com/}{applications},
        \href{http://www.scipy.org/Cookbook/OptimizationDemo1}{scientific computations},
        \href{http://www.prairiegames.com/}{games}, ...
\end{itemize}
\end{frame}

\begin{frame}[fragile]{Python implementations}
There are several interpreters for the Python language:
\begin{itemize}
\item CPython, the original one (written in C)
\item Jython, written in Java {\small (can use Java libraries directly)}
\item IronPython, written in C\verb/#/
\item PyPy, written in Python itself
\item Stackless Python, with some adventurous enhancements
\end{itemize}
\bigskip
We'll be using CPython in this course.
\end{frame}

\begin{frame}{Version confusion}
There are two current versions of Python:
\begin{itemize}
\item 2.7, which is what we are going to use
\item 3.1, the “future” version \\ {\small Many improvements, but many libraries aren't yet compatible}
\end{itemize}
The school computers use version 2.4 {\small (rather old, but it still works)}

\end{frame}

\section{Installing Python}

\begin{frame}{Installing}
Windows
\begin{itemize}
\item Go to \href{http://python.org/downloads}{python.org/downloads} and download the installer
\item Install it
\item Check if python's in your PATH
\end{itemize}
Linux
\begin{itemize}
\item You already have it
\item If you don't, use your package manager
\end{itemize}
Mac OS X
\begin{itemize}
\item You also already have it
\end{itemize}
\bigskip
\end{frame}

\begin{frame}[fragile]{The Interactive Shell}
Check your installation on a terminal (command prompt, console):
\begin{verbatim}$ python
>>> 1 + 1
2
>>> print 'hello world'
hello world
>>> x = 6
>>> y = 7
>>> x * y
42
>>> exit()  # or Ctrl+D (Mac/Linux) / Ctrl+Z (Win)\end{verbatim}\gobble{ % This is for doctest:
Traceback (most recent call last):
  File "/usr/lib/python2.6/site.py", line 337, in __call__
  raise SystemExit(code)
SystemExit: None

}
\end{frame}

\begin{frame}{Setting up an editor}
\begin{itemize}
\item Many editors have Python syntax highlighting and indentation modes or plugins, find them and use them!
\item Set your editor to indent using 4 spaces
\bigskip
\item Now, you're ready for your first program!
\end{itemize}
\end{frame}

\end{document}
