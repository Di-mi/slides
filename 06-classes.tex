\documentclass{pyslides}

\title{Classes}
\pyslidenumber{6}
\date{November 2010}

\newcommand\im[1]{\par\vspace{3pt}\hspace{0ex}\rlap{\tt #1}\hspace{3.5cm}}
\newcommand\imm[1]{\par\vspace{3pt}\hspace{0ex}\rlap{#1}\hspace{4.5cm}}
\newcommand\nim{\par\hspace{3.5cm}}

%%%%%%%%%%%%%%%%%%%%%%%%%%%%%%%%%%%%%%%%%%%%%%%%%%%
\begin{document}

\begin{frame}\titlepage\end{frame}

\section{Classes}

\begin{frame}[fragile]{Example program}
Download \texttt{fib.py} from the course page:
\bigskip

\input "|./highlight.py 'samples/06fib.py' fontsize=!scriptsize"
\end{frame}

\begin{frame}[fragile]{What's a class?}
\input "|./highlight.py 'samples/06fib.py' fontsize=!small,lastline=1"

Classes are a mechanism for defining your own objects.

The \texttt{Fib} class defines objects that are able to give Fibonacci numbers:

\input "|./highlight.py 'samples/06usefib.txt' fontsize=!small no"
\end{frame}

\begin{frame}[fragile]{Methods}
\input "|./highlight.py 'samples/06fib.py' fontsize=!small,firstline=6,firstnumber=6,lastline=9"

\emph{Methods} of a class are defined as functions that receive the current \emph{instance} as their first argument (usually named \texttt{self}).

\bigskip

\texttt{\magic{init}} is a special method that is called when the class is instantiated: a \emph{constructor}

\end{frame}

\begin{frame}[fragile]{Methods}
\input "|./highlight.py 'samples/06fib.py' fontsize=!small,firstline=11,firstnumber=11,lastline=16"

\texttt{next} is perfectly normal method that changes instance attributes, and returns a value.

\bigskip

When calling a method, leave the \texttt{self} argument out:

\input "|./highlight.py 'samples/06usefib.txt' fontsize=!small,firstline=4,lastline=5 no"

\end{frame}

\begin{frame}[fragile]{Magic methods}
There are many “magic methods” that Python calls “behind your back”, like \magic{init} – for example these:

\imm{\magic{init}(self)}        The constructor
\imm{\magic{str}(self)}         \texttt{str(instance)}
\imm{\magic{nonzero}(self)}     \texttt{bool(instance)}
\imm{\magic{getattr}(self, a)}  \texttt{instance.a\_value}
\imm{\magic{getitem}(self, a)}  \texttt{instance[a]}
\imm{\magic{call}(self, *args)} \texttt{instance(*args)}
\imm{\magic{eq}(self, other)}   \texttt{instance == other}
\imm{\magic{add}(self, other)}  \texttt{instance + other}
\end{frame}

\begin{frame}[fragile]{Superclasses}
\input "|./highlight.py 'samples/06fib.py' fontsize=!small,lastline=1"

Classes are \emph{derived} from other classes, \emph{inheriting} their behavior.

\bigskip

If you want nothing special, inherit from \texttt{object}.

If you inherit from something else, then the new class (subclass) will, by default, behave like the one inherited from (superclass).

\bigskip

You can also inherit from multiple classes: separate them by commas.

\end{frame}

\begin{frame}[fragile]{A pretty-printing list}
\input "|./highlight.py 'samples/06inheritlist.txt' fontsize=!small no"
\end{frame}

\section{Namespaces}

\begin{frame}[fragile]{Namespaces}
Understanding the rest of these slides are not necessary to use Python.

\bigskip

{\small (If you want to study this later by yourself, look at \href{http://docs.python.org/tutorial/classes.html}{docs.python.org/tutorial/classes.html}.)}
\end{frame}

\begin{frame}[fragile]{Namespaces}

In Python, a \emph{namespace} is basically a dictionary that maps variable names to values.

Every module has a namespace, called the \emph{module-level} or \emph{global} namespace.

Variables and functions not defined in functions (or classes) live there.

Objects in the global namespace are accessible as \emph{attributes} of the module.

There is also a \emph{built-in} namespace that all modules share.

\end{frame}


\begin{frame}[fragile]{Function namespaces: Assigning}

Every function has its own namespace. The function's arguments are automatically added there.

\emph{Assigning} to a variable assigns to the function's namespace.

\bigskip

The \PY{k}{\tt global} keyword can be used to force assignments to the global (module-level) namespace.
\end{frame}

\begin{frame}[fragile]{Function namespaces: Using}

When you \emph{use} a variable, Python looks in:
\begin{itemize}
\item The current function's namespace
\item The enclosing functions' namespace(s)
\item The global namespace
\item The built-in namespace
\end{itemize}

\bigskip

\input "|./highlight.py 'samples/06funcns.py' fontsize=!small no"
\end{frame}


\begin{frame}[fragile]{Closures}
\input "|./highlight.py 'samples/06funcinfunc.py' fontsize=!small no"
\end{frame}

\section{How classes work}

\begin{frame}[fragile]{Classes}
Creating a~\emph{class} also creates an object with its own namespace.

\bigskip

\input "|./highlight.py 'samples/06class1.py' fontsize=!small no"

\end{frame}

\begin{frame}[fragile]{Instances}
\emph{Calling} a~class creates an \emph{instance} of the class.

Instance attributes are independent of the class attributes.

But when there is no instance attribute of some name, the class' one is used.

\bigskip

\input "|./highlight.py 'samples/06instantiation.py' fontsize=!small no"

\end{frame}

\begin{frame}[fragile]{Instances attributes}
In this example, each {\tt exampleInstance.a} on line 6 is in a different namespace.

\bigskip

\input "|./highlight.py 'samples/06instantiaceex.py' fontsize=!small"

\end{frame}


\begin{frame}[fragile]{Methods}
Functions defined in a class are called \emph{methods}.

If a class' method is looked up by an instance's attribute access, the method is \emph{bound} to the instance.

\input "|./highlight.py 'samples/06methods.txt' fontsize=!small no"

\end{frame}

\begin{frame}[fragile]{Bound Methods}
A \emph{bound method} has its first argument set to the instance it is bound to.

\input "|./highlight.py 'samples/06boundmethod.txt' fontsize=!small no"

\end{frame}

\begin{frame}[fragile]{Unbound Methods}
An \emph{unbound method} needs an instance as its first argument.

\input "|./highlight.py 'samples/06unboundmethod.txt' fontsize=!small no"

\end{frame}

\begin{frame}[fragile]{Superclasses}
A \emph{superclass} of a class is where an attribute is searched for if it's not found in the class.

\input "|./highlight.py 'samples/06superclass.txt' fontsize=!small no"

\end{frame}

\begin{frame}[fragile]{Metaclasses}
Metaclasses are \emph{classes of classes}.

Go read about them if this was not enough of an intelectual challenge.
\end{frame}


\end{document}
