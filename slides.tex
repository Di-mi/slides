\documentclass[20pt]{beamer}
\usepackage{fontspec}
\usepackage{amsfonts}
\usepackage{tangocolors}

\usepackage{tikz}
\usetikzlibrary{shapes}

\newcommand\sk{\par\bigskip\bigskip\par}
\newcommand\wh[1]{\only<#1>{\color{white}}}
\newcommand\tx[2]{\alt<#1>{\textcolor{ta3gray}}{\textcolor{ta2gray}}{\uncover<#1->{#2}}}
\newcommand\rd[2]{\alt<#1>{\textcolor{tascarletred}}{\textcolor{ta3aluminium!50!tascarletred}}{\uncover<#1->{#2}}}

\begin{document}
\fontspec{Fertigo Pro}

\begin{center}
\title{SQLAlchemy}
\author{Petr Viktorin}
\date{\today}

\frame{
    \sk
    \textcolor{ta2gray}{\tiny Povídání o}\\
    \textcolor{ta3gray}{SQL}\textcolor{tascarletred}{Alchemy}
    \sk\sk
    \textcolor{ta2gray}{Petr Viktorin}\\[-0.25cm]
    \textcolor{ta2gray}{\tiny @encukou}
    \sk
    \textcolor{ta2gray}{\tiny lednové PyVo 2010 – NoSQL}\\[-0.25cm]
    \pause
    \textcolor{ta3gray}{\tiny (co tu vůbec dělám?)}
}

\frame{
    \rd{1}{SQL?}
    \sk
    \tx{1}{To je taková ta zastaralá technologie, ne?}
    \sk

    \rd{2}{1970}
    \tx{2}{– relační databáze}

    \rd{3}{1979}
    \tx{3}{– SQL}
}

\frame{
    \rd{1}{Problémy}
    \tx{1}{SQL}
    \sk
    \tx{2}{Kompatibilita,}
    \rd{2}{standardizace}
    \tx{2}{..?}
    \sk
    \tx{3}{Deklarativní jazyk,}
    \rd{3}{textový}
    \tx{3}{formát}
    \sk
    \tx{4}{Pythoní}
    \rd{4}{DBAPI}
    \tx{4}{(PEP 249)}
    \tx{4}{má stejné problémy}
}

\frame{
    \begin{tikzpicture}
    \alt<1>{
        \node[forbidden sign,text width=4em, text centered,draw=tascarletred, line width=0.25cm] {\huge \tx{1}{SQL}}; }{
        \node[forbidden sign,text width=4em, text centered,draw=ta3aluminium!50!tascarletred, line width=0.25cm] {\huge \tx{1}{SQL}}; }
    \end{tikzpicture}
    \sk
    \tx{1}{Nechci psát}
    \rd{1}{čisté}
    \tx{1}{SQL}

    \rd{2}{Ale}\\
    \tx{3}{chci používat \rd{3}{relační databáze}!}
}

\frame{
    \tx{1}{
        “\rd{1}{Disproving the myth} of ‘the best database layer is
        the one that makes the database invisible’ is a
        primary philosophy of SA.
    }

    \tx{2}{
        If you don't want to deal
        with SQL, then there's little point to using a
        database in the first place.”
    }
}

\frame{
    \tx{1}{SQLAlchemy}
    \rd{1}{Core}
    \sk
    \tx{2}{· Definice tabulek}
    \sk
    \tx{2}{· Dotazování}
}

\begin{frame}[fragile]
    \tx{1}{Definice tabulek}

        \color{ta3gray}
    \tiny
        \begin{verbatim}
from sqlalchemy import Table, Column, Integer, String,
        MetaData, ForeignKey
metadata = MetaData()

users = Table('users', metadata,
    Column('id', Integer, primary_key=True),
    Column('name', String),
    Column('fullname', String),
)

addresses = Table('addresses', metadata,
  Column('id', Integer, primary_key=True),
  Column('user_id', None, ForeignKey('users.id')),
  Column('email_address', String, nullable=False)
 )
        \end{verbatim}
\end{frame}

\begin{frame}[fragile]
    \tx{1}{Dotazování}

        \color{ta3gray}
    \tiny
        \begin{verbatim}
from sqlalchemy import create_engine
from sqlalchemy.sql import select

engine = create_engine('sqlite:///:memory:', echo=True)

conn = engine.connect()

s = select([users])
result = conn.execute(s)

for row in result:
    print row\end{verbatim}%
        \color{ta2gray}%
        \begin{verbatim}(1, u'jack', u'Jack Jones')
(2, u'wendy', u'Wendy Williams')
(3, u'fred', u'Fred Flintstone')
(4, u'mary', u'Mary Contrary')
        \end{verbatim}
\end{frame}

\begin{frame}[fragile]
    \tx{1}{Dotazování}

        \color{ta3gray}
    \tiny
        \begin{verbatim}
s1 = select([users])
s1 = s1.filter(users.name == 'fred')

s2 = select([users])
s2 = s2.filter(users.id >= 3)

s3 = select([users.c.name, users.c.fullname])

s4 = select([users, addresses],
        users.c.id==addresses.c.user_id)


\end{verbatim}%
\end{frame}

\begin{frame}[fragile]
    \tx{1}{Dotazování}

        \color{ta3gray}
    \tiny
        \begin{verbatim}

s = select([(users.c.fullname + ", " +
            addresses.c.email_address).label('title')],
        and_(
            users.c.id==addresses.c.user_id,
            users.c.name.between('m', 'z'),
           or_(
              addresses.c.email_address.like('%@aol.com'),
              addresses.c.email_address.like('%@msn.com')
           )
        )
    )\end{verbatim}%
    \tx{2}{Dotazy a výrazy jsou \rd{2}{objekty} – dá se s nimi dál pracovat}
\end{frame}

\frame{
    \tx{1}{Ale počkat, nemělo SQLAlchemy být \rd{1}{ORM}?}
    \sk
    \tx{2}{Object-Relational Mapper}
}

\frame{
    \tx{1}{Architektura SQLAlchemy}
    \sk
    \tiny
    \begin{tikzpicture}[x=0.5cm,y=0.5cm]
        \draw[tascarletred] (20.5, -2.5) rectangle (0, 4) node[rectangle, anchor=north west]{\tx{1}{SQLAlchemy} \rd{1}{Core}};
        \draw[tascarletred] (20.5, 4.5) rectangle (0, 8.5) node[rectangle, anchor=north west]{\tx{1}{SQLAlchemy} \rd{1}{ORM}};
        \node[rectangle,rounded corners,draw, anchor=south west,text width=9.5cm,align=center,minimum height=1cm] at (0.5,5) {\centering Object Relational Mapper};
        \node[rectangle,rounded corners,draw, anchor=south west,text width=2cm,align=center,minimum height=1cm] at (0.5,0.5) {\centering Schema, Types};
        \node[rectangle,rounded corners,draw, anchor=south west,text width=2cm,align=center,minimum height=1cm] at (5.5,0.5) {\centering Expression Language};
        \node[rectangle,rounded corners,draw, anchor=south west,text width=4.5cm,align=center,minimum height=1cm] at (10.5,0.5) {\centering Engine};
        \node[rectangle,rounded corners,draw, anchor=south west,text width=2cm,align=center,minimum height=1cm] at (10.5,-2) {\centering Connection Pool};
        \node[rectangle,rounded corners,draw, anchor=south west,text width=2cm,align=center,minimum height=1cm] at (15.5,-2) {\centering Expression Dialect};
        \node[rectangle,rounded corners,draw, anchor=south west,text width=2cm,align=center,minimum height=1cm] at (15.5,-5) {\centering DBAPI};
    \end{tikzpicture}
}

\begin{frame}[fragile]
    \tx{1}{Deklarativní model}

        \color{ta3gray}
    \tiny
        \begin{verbatim}
from sqlalchemy import Column, Integer, String, ForeignKey
from sqlalchemy.ext.declarative import declarative_base
metadata = MetaData()

Base = declarative_base()

class User(Base):
    __tablename__ = 'users'
    id = Column(Integer, primary_key=True)
    name = Column(String)
    fullname = Column(String)

class Address(Base):
    __tablename__ = 'users'
    id = Column(Integer, primary_key=True),
    user_id = Column(None, ForeignKey('users.id')),
    email_address = Column(String, nullable=False)
    user = relationship("User",
            backref=backref('addresses', order_by=id))

        \end{verbatim}
\end{frame}

\begin{frame}[fragile]
    \tx{1}{Unit of Work}

        \color{ta3gray}
    \tiny
        \begin{verbatim}
from sqlalchemy.orm import sessionmaker
Session = sessionmaker(bind=engine)

session = Session()

ed_user = User('ed', 'Ed Jones', 'edspassword')
session.add(ed_user)

session.commit()

        \end{verbatim}
\end{frame}

\begin{frame}[fragile]
    \tx{1}{Dotazování}

        \color{ta3gray}
    \tiny
        \begin{verbatim}

session.select(User).filter(User.name == 'ed')
        \end{verbatim}
    \tx{1}{Opět plné možnosti \rd{1}{SQL}}
\end{frame}

\begin{frame}[fragile]
    \tx{1}{\rd{1}{ORM} je nadstavba nad \rd{1}{Core}}

        \color{ta3gray}
    \tiny
        \begin{verbatim}

>>> User.__table__
Table('users', MetaData(None),
            Column('id', Integer(), table=<users>,
                    primary_key=True, nullable=False),
            Column('name', String(), table=<users>),
            Column('fullname', String(), table=<users>),
            Column('password', String(), table=<users>),
                    schema=None)

        \end{verbatim}
\end{frame}

\frame{
    \tx{1}{“The \rd{1}{Tao} of SQLAlchemy”}
    \sk
    \tx{1}{
        SQL databases behave less like \rd{1}{object collections} the more \rd{1}{size and performance} start to matter
        \\[0.25cm]
        Object collections behave less like \rd{1}{tables and rows} the more \rd{1}{abstraction} starts to matter
    }
}

\frame{
    \tx{1}{“The \rd{1}{Tao} of SQLAlchemy”}
    \sk
    \tx{1}{Neschovávat detaily, ale \rd{1}{automatizovat}}
    \sk
    \tx{2}{Uživatel má plnou \rd{1}{kontrolu}}
    \sk
    \tx{3}{PEP\rd{1}{20} (import this)}
}

\frame{
    \tx{1}{Díky za pozornost!}
    \sk
    \rd{1}{Dotazy?}
}

\frame{
    \tx{1}{Zdroje}
    \tiny
    \sk
    \url{http://www.sqlalchemy.org}
    \sk
    \url{http://spyced.blogspot.com/2009/05/belated-2009-introduction-to-sqlalchemy.html}
    \sk
    \url{http://www.sqlalchemy.org/blog/2011/09/19/sqlalchemy-at-pygotham/}
}

\end{center}
\end{document}

